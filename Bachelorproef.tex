\documentclass[a4paper]{article}
\usepackage{amsmath,amssymb}
\usepackage[dutch]{babel}
\usepackage{graphicx}
\usepackage{float}
\usepackage[utf8]{inputenc}
\usepackage{url}

\title{Voltage cotrolled liquid crystal lens based on solvent-assisted nanolithography.}
\author{Wout Mens, Viktor Op de Beeck}

\begin{document}
\maketitle
\section{Abstract}
At an age of about 40 years, the risk of cataract starts increasing every decade. Traditionally the troubled lens of a patient suffering from cataract is surgically removed and replaced by an artificial lens. The lenses used for this are usually single- or multi-focus. In this project we will build and investigate the characteristics of an accommodating liquid crystal lens. The dioptric strength of this lens will be controlled by applying a voltage to the lens, which changes the orientation of the liquid crystals. The focus of this project will be aimed towards the construction of such a lens, its characterization and compare the experimental data with model based calculations. Lenses with a variable dioptric strength already exist and are available, these lenses are however not based on liquid crystals but on electrowetting.

\section{Introduction}
As described in the abstract this project concerns a lens based on liquid crystals, more specifically nematic liquid crystals. These crystals have a fixed orientation, the orientation used for the lens is the planar orientation (parallel to the lens and glass). This can be achieved through a process called rubbing, where a felt cloth is rubbed on a polymer layer on the lens. The method that will be used in this project however is based on the nanogrooves found in a DVD. These grooves will be imprinted on a polyvinyl alcohol layer. The orientation of these crystals is of utmost importance because the orientation of the crystals determines the phase-shift and the polarisation (and thus the focus in a plane) of light travelling through the lens. Since this orientation determines the polarization of light, we want to be able to control this orientation in order to achieve a lens with a variable dioptric strength. The orientation can be controlled by placing the crystals in an electric field, which will be achieved by applying a voltage to the lens. This voltage will be that of an alternating current as a direct current would draw all charges such as ions to one side which would disrupt the electric field. Applying a larger voltage to the lens would result in a larger penetration depth of an orientation change, because of this the curvature of the lens can not be too great for then there would be crystals whose orientation never changes. In order to achieve this potential on the lens a conducting layer has to be integrated in the lens. This layer is integrated beneath the polymer layer in which the nanogrooves are imprinted. The materials used for this conducting layer ideally satisfy 2 characteristics. They should be transparent, so that they don't alter the produced image and they should also be conducting materials. A material that is frequently used is ITO (IndiumTinOxide) this material is transparent and a conductor, another material that can be used is a very thin layer of aluminum, this is a good conductor but it is not completely transparent, hence very thin layers have to be used. Another challenge that arises is the connection of the electrodes, in order to do this a conducting material is vaporized around the edge of the lens. Furthermore the entire lens has to be held together, by for instance glue. Likewise a challenge that is encountered in characterising the lens is the large uncertainty when studying large focal lengths, it is hard to tell at exactly what distance the produced image is the sharpest. Criteria for the sharpness of an image can be introduced for this purpose. An image of a black line is used, this image has a darkness profile of a delta function. Then the darkness profile of the image produced by the lens, which approaches a Gaussian when the thickness of the line tends to zero, can be studied and the width of the produced darkness profile can be used as a parameter for the sharpness of the image. 
\end{document}